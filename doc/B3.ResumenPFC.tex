%%%%%%%%%%%%%%%%%%%%%%%%%%%%%%%%%%
% Página de resumen del proyecto %
%%%%%%%%%%%%%%%%%%%%%%%%%%%%%%%%%%

\thispagestyle{empty}
\begin{spacing}{1}

\begin{center}
	\Large \sffamily
	Universidad de Málaga\\
	Escuela Técnica Superior de Ingeniería de\\
	Telecomunicación
\end{center}

\bigskip

\begin{center}
	\Huge\scshape
	\pfctitlename
\end{center}

\bigskip

\begin{center}
	\textbf{REALIZADO POR}\\
	\textsf{\pfcauthorname}
\end{center}

\medskip

\begin{center}
	\textbf{DIRIGIDO POR}\\
	\textsf{\pfctutorname}
\end{center}

\vfill

\begin{minipage}{\textwidth}
\textbf{Dpto. de:} Ingeniería de Comunicaciones (IC)

\medskip

\textbf{Palabras clave:} Bluetooth, Android, aplicación, localización, web, corazón

\medskip

\textbf{Titulación:} Ingeniería de Telecomunicación

\medskip

\textbf{Resumen:}	
	El presente proyecto pretende realizar una prueba de concepto sobre el diseño e implementación de un sistema de monitorización del ritmo cardíaco, a través del conjunto de dos aplicaciones, una aplicación Android que se conectará y recibirá datos del sensor mediante el protocolo Bluetooth 4.0 de baja energía, transmitiendo dichos datos a un servidor web y una aplicación web, que usará una conexión web socket para mostrar los datos en tiempo real

\begin{center} Málaga, \today\end{center}
\end{minipage}
\end{spacing}
\blankpage